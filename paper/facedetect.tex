\section*{Composition and Face Detection}

We analyze film composition at two levels of abstraction: the \emph{frame} and the \emph{sequence}.  A frame is a single image produced by a camera in the film.  A sequence is a series of camera shots, where each shot is an unbroken series of frames produced by a single camera.  Frame composition and sequence structure provide a sufficient basis upon which to begin analysis, are explicitly defined, and are directly related to the work of the cinematographer.

To analyze frame composition and sequence structure we first classify each frame and then use patterns in the sequence of frame classes to determine sequential structure.  We restrict our classification of frame composition in this work to frame orientations which 

\subsection{Frame Composition}

We restrict our description of the frame composition to population, zoom
and position of actors in the frame. Other visual elemements, such as
non-human points of interest (e.g. a car in the midst of
a chase sequence) can be as or more important than actors in a given frame;
 however, actors are at the core of most motion-pictures, and are therefore a good point of
 entry for better automatic understanding frame composition.

To detect the population, zoom and position of actors in each frame we 
use a state-of-the-art face detector~\cite{?}.  

\subsubsection{Population}
The population of a frame is defined as the number of faces visible. We consider a face to be valid whether it is an actor, a photograph, a movie-within-a-movie or some other indirect face. A face is also valid regardless of minor occlusions (e.g. hair or partial masks) and orientation (e.g. face is oriented away from the camera). A face is invalid if it is non-human (e.g. animals, monsters, drawings, or CGI renderings). 

%--------------------------------------------------------------------------------

\subsubsection{Zoom}
Zoom is the perceived distance between camera and actor. This is typically classified by how much of the actor is visible within the shot. Figure \ref{fig:zoomClass} shows the seven commonly used zoom classifications. We stick to the traditional nomenclature of 'shot', though we are classifying frames. The classifications  are, in descending order of 'closeness':


%%%%%%%%%%%%%%%%%%%%%%%%%%%%%%
\textbf{Extreme Close-Up - ECU} In these frames, the camera is so close to the actor that the audience cannot see the actor's entire face. This creates a frame in which the actor's face exceeds the size of the frame.

\textbf{Close-Up - CU} In a Close-Up, the viewer can see the actors entire face, and perhaps their neck and trapezius, however the shoulders are typically not in the frame. 

\textbf{Medium Close-Up - MCU} Between a medium shot and a close-up, these frames contain the actor's entire face as well as their neck, shoulders, and clavicular region. 

\textbf{Medium Shot - MS} A Medium Shot shows the actor from the abdomen and above. 

\textbf{Medium Long Shot - MLS} These frames contain most, but not all of an actor's body. The shots typically contain some or all of the leg, but will not show the feet. A common location to cut-off the actor in these shots is at the knees. Cutting at mid-thigh, the knees, or mid-shin is considered more visually appealing than cutting at the ankles \cite{arijon_grammar_1991}.

\textbf{Long Shot - LS}In a Long Shot, the actor's entire body, from head to toe, is visible. In these frames, the actor is generally still in the forefront of the frame. 

\textbf{Extreme Long Shot - ELS} In these shots, the actor is visible, but at a significant distance from the camera. 

%%%%%%%%%%%%%%%%%%%%%%%%%%%%%%

We assume that the proportion between the height of an actor's head and the height of their body is relatively consistent across actors. We define the 7 types of zoom by the ratio of the face height, $f$, to the frame height, $h$. By taking measurements of pre-classified frames, we computed the following $f$:$h$ ratios for each class of zoom:


%%%%%%%%%%%%%%%%%%%%%%%%%%%%%%
\begin{center}
  \begin{tabular}{ l | l l}
    Shot Name & Abbrv. & Range \\ \hline
    Extreme Close-Up & ECU & $f > h$ \\ 
    Close-Up & CU & $h \geq f > 0.6h$ \\ 
    Medium Close-Up & MCU & $0.6h \geq f > 0.3h$ \\ 
    Medium Shot & MS & $0.3h \geq f > 0.2h$ \\ 
    Medium Long Shot & MLS & $0.2h \geq f > 0.1h$ \\ 
    Long Shot & LS & $0.1h \geq f > 0.02h$ \\ 
    Extreme Long Shot & ELS & $0.02h \geq f$
    \label{tab:zoomTypes}
  \end{tabular}
\end{center}
%%%%%%%%%%%%%%%%%%%%%%%%%%%%%%


These ratios are applied to the bounding boxes produced by face detectors. The height of the bounding box our $f$ and the vertical resolution of the frame our $h$. Note that in the case of frames with population $> 1$, we tag the frame based on the face closest to the camera. 
%--------------------------------------------------------------------------------
\subsubsection{Positioning}
Frame composition is also defined by the position of faces in the frame. Texts on cinematography frequently divide a frame horizontally into 2, 3, or 4 parts and vertically into 2 or 3 parts. We divide the frame into a $3\times 3$ grid. The vertical classifications are top, middle, \& bottom. The horizontal classifications are left, center, \& right. These classifications are considered in isolation or combinatorially to create, for example, top-right and middle-center. This classification has the advantage of clear perceptual salience. While it is difficult to discern whether a face is in the left quarter of a frame compared to the left third of a frame, it is much easier to determine whether it is centered, in the left third, or in the right third. Edge cases, such as frames composed using the `rule of thirds', are difficult for a human to classify; however, the $3\times 3$ grid offers a good compromise between precision and visual salience. A frame is given a tag corresponding to the grid cell containing the centroid of the largest face in the frame. Figure \ref{fig:gridLabels} shows the tags. %--------------------------------------------------------------------------------


\begin{figure}
  \begin{center}
  \begin{tabular}{ c |c |c }
    \large{TL} & \large{TC} & \large{TR}\\
    Top Left & Top Center & Top Right \\
    \hline
    \large{ML} & \large {MC} & \large {MR }\\
    Middle Left & Middle Center & Middle Right \\
    \hline
    \large{BL} & \large {BC} & \large {BR }\\ 
    Bottom Left & Bottom Center & Bottom Right
    \label{fig:gridLabels}
  \end{tabular}
  \caption{The grid labels and their abbreviations.}
\end{center}
\end{figure}

 