\section*{Introduction}

Much like authors and editors of written language, cinematographers and film editors have their own unique style.
Modern cinematographers use a well-defined visual language to create motion-pictures. Our goal is to better understand the use of cinematographic techniques in defining a motion-picture's visual style. What types of shots are most common? Are certain shots arranged in predictable ways across many motion-pictures?

In this work we focus on automatically classifying frame composition and sequential structure in film.  To do so we first classify the composition of individual frames based on the number of actors and placement of each actor in the frame.  Then, we analyze the the sequential structure of a film by mining patterns in the frame classes determined in the previous step.  The classes and sequential structures recovered during this process closely match with the terminology used by professional cinematographers~\cite{arijon_grammar_1991}.  

We finally recognize that we can better understand film composition by also recognizing a film's shots (transitions from one camera to another), and we implement two state-of-the-art approaches for camera shot boundary detection.

%%% Local Variables:
%%% mode: latex
%%% TeX-master: "finalpaper"
%%% End:
