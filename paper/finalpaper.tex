%%%%%%%%%%%%%%%%%%%%%%%%%%%%%%%%%%%%%%%%%
% Thin Sectioned Essay
% LaTeX Template
% Version 1.0 (3/8/13)
%
% This template has been downloaded from:
% http://www.LaTeXTemplates.com
%
% Original Author:
% Nicolas Diaz (nsdiaz@uc.cl) with extensive modifications by:
% Vel (vel@latextemplates.com)
%
% License:
% CC BY-NC-SA 3.0 (http://creativecommons.org/licenses/by-nc-sa/3.0/)
%
%%%%%%%%%%%%%%%%%%%%%%%%%%%%%%%%%%%%%%%%%

%----------------------------------------------------------------------------------------
%	PACKAGES AND OTHER DOCUMENT CONFIGURATIONS
%----------------------------------------------------------------------------------------

\documentclass[a4paper, 11pt]{article} % Font size (can be 10pt, 11pt or 12pt) and paper size (remove a4paper for US letter paper)

\usepackage[protrusion=true,expansion=true]{microtype} % Better typography
\usepackage{graphicx} % Required for including pictures
\usepackage{wrapfig} % Allows in-line images
\usepackage{gensymb}

\usepackage[font={small,it}]{caption}
\usepackage[font={scriptsize,it}]{subcaption}

\captionsetup{justification=centering}

\usepackage{mathpazo} % Use the Palatino font
\usepackage[T1]{fontenc} % Required for accented characters
\linespread{1.05} % Change line spacing here, Palatino benefits from a slight increase by default

\makeatletter
\renewcommand\@biblabel[1]{\textbf{#1.}} % Change the square brackets for each bibliography item from '[1]' to '1.'
\renewcommand{\@listI}{\itemsep=0pt} % Reduce the space between items in the itemize and enumerate environments and the bibliography

\renewcommand{\maketitle}{ % Customize the title - do not edit title and author name here, see the TITLE block below
\begin{centering} % Right align
{\huge\@title} % Increase the font size of the title

\vspace{50pt} % Some vertical space between the title and author name

{\large\@author} % Author name
%\\\@date % Date

\vspace{40pt} % Some vertical space between the author block and abstract
\end{centering}
}

%----------------------------------------------------------------------------------------
%	TITLE
%----------------------------------------------------------------------------------------

\title{\textbf{Deep Cuts}\\ % Title
\Large{shot detection for cinematographic analysis}} % Subtitle

\author{\textsc{Rachel Albert\\ Alex Hall\\ Jonathan Harper\\ Amy Pavel} % Author
\\{\textit{\\CS280 Final Project, Spring 2015}}} % Institution


%\date{\today} % Date

%----------------------------------------------------------------------------------------

\begin{document}

\maketitle % Print the title section

%----------------------------------------------------------------------------------------
%	ABSTRACT AND KEYWORDS
%----------------------------------------------------------------------------------------

%\renewcommand{\abstractname}{Summary} % Uncomment to change the name of the abstract to something else

%\begin{abstract}
%Text goes here
%\end{abstract}
%
%\hspace*{3,6mm}\textit{Keywords:} lorem , ipsum , dolor , sit amet , lectus % Keywords
%
%\vspace{30pt} % Some vertical space between the abstract and first section

%----------------------------------------------------------------------------------------
%	ESSAY BODY
%----------------------------------------------------------------------------------------

\section*{Introduction}

Short introduction goes here.

%------------------------------------------------

\section*{Background \& Significance}

Literature review and short summary of our contribution goes here.

%--------------------------------------------------

\section*{Style and Face Detection}

Description of Alex's work goes here.


\subsection*{Methods}

Briefly describe the algorithms used.

\subsection*{Results}

Show the results.

%------------------------------------------------

\section*{Shot Detection}

Short summary of why we did shot detection (transition from face detection) goes here.

\subsection*{Adaptive Thresholding}

Briefly describe thresholding method and show the results.

\subsubsection*{Random Forests}

Briefly describe random forests method and show the results.

\subsubsection*{CNN}

Briefly describe CNN method and show the results.

%------------------------------------------------

\section*{Summary \& Conclusion}

Short summary and conclusion goes here.

%----------------------------------------------------------------------------------------
%	BIBLIOGRAPHY
%----------------------------------------------------------------------------------------

\bibliographystyle{unsrt}

\bibliography{sample}

%----------------------------------------------------------------------------------------

\end{document}